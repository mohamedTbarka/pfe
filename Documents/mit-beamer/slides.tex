\documentclass{beamer}
%\documentclass[handout,t]{beamer}

\batchmode
% \usepackage{pgfpages}
% \pgfpagesuselayout{4 on 1}[letterpaper,landscape,border shrink=5mm]

\usepackage{amsmath,amssymb,enumerate,epsfig,bbm,calc,color,ifthen,capt-of,float}
\usepackage{algorithm2e}
\usepackage{algorithmic}


\usetheme{Berlin}
\usecolortheme{mit}

\title{Hierarchical Leader Election Algorithm}
\author{Mohamed Tbarka}
\date{\today}
\pgfdeclareimage[height=0.5cm]{mit-logo}{mit-logo.pdf}
\logo{\pgfuseimage{mit-logo}\hspace*{0.3cm}}

\AtBeginSection[]
{
  \begin{frame}<beamer>
    \frametitle{Outline}
    \tableofcontents[currentsection]
  \end{frame}
}
\beamerdefaultoverlayspecification{<+->}
% -----------------------------------------------------------------------------
\begin{document}
% -----------------------------------------------------------------------------

\frame{\titlepage}

\section[Outline]{}
\begin{frame}{Outline}
  \tableofcontents
\end{frame}

% -----------------------------------------------------------------------------
\section{Introduction}
\subsection{Election Algorithms}
\begin{frame}{The Bully Algorithm}
  As a first example, consider the bully algorithm devised by Garcia-Molina (1982). When any process notices that the coordinator is no longer responding to requests, it initiates an election. A process, P, holds an election as follows:
  \pause
  \begin{itemize}
    \item <2-> P sends an ELECTION message to all processes with higher numbers.
    \item <3-> If no one responds, P wins the election and becomes coordinator.
    \item <4-> If one of the higher-ups answers, it takes over. P's job is done.

  \end{itemize}
\end{frame}
\begin{frame}{A Ring Algorithm}
  \begin{figure}[h]
  	\centering
  	\includegraphics[width=0.7\linewidth]{../../Desktop/figure_3}

  	\label{fig:figure1}
  \end{figure}
  
\end{frame}

\begin{frame}{Career}
  \begin{itemize}
    \item<1-> Chacvirman of the American National Standards Institute 
    \item<2-> Served as president of the International Organization for Standardization from 2003 to 2005.
  \end{itemize}
\end{frame}
% -----------------------------------------------------------------------------
\section{Preliminaries}
\subsection{Where can I learn more?}
\begin{frame}{Questions and Answers}
Want to know more?

\begin{itemize}
	\item Browse \url{http://web.mit.edu/smoot/history.htm}.
	\item Smoot's Legacy \url{http://alum.mit.edu/news/AlumniNews/Archive/smoots_legacy}.
	\item Smoot Salute! \url{http://web.mit.edu/spotlight/smoot-salute}.
\end{itemize}

\end{frame}

% -----------------------------------------------------------------------------
\section{Hierarchical Leader Election Algorithm}
\subsection{The Pseudo-code}

\begin{frame}{The Pseudo-code}

\begin{algorithm}[!H]
	\begin{algorithmic}[1]
		\FOR{$i=1$ to $N$}
		\FOR{$j=1$ to $JJJJ$}
		\STATE $energy[i*JJJ+j] =$ \\
		$ interpolate(AAA[i*JJJ+j], ZZZ)$
		\ENDFOR
		\ENDFOR
	\end{algorithmic}
	\caption{pseudocode for the calculation of }
	\label{alg:seq}
\end{algorithm}
\end{frame}

% -----------------------------------------------------------------------------
\section{Correctness}

% -----------------------------------------------------------------------------

\section{Implementation}

% -----------------------------------------------------------------------------
\section{title}
% -----------------------------------------------------------------------------
\section{Conclusions}
\subsection{Where can I learn more?}
\begin{frame}{Questions and Answers}
  Want to know more?

  \begin{itemize}
    \item Browse \url{http://web.mit.edu/smoot/history.htm}.
    \item Smoot's Legacy \url{http://alum.mit.edu/news/AlumniNews/Archive/smoots_legacy}.
    \item Smoot Salute! \url{http://web.mit.edu/spotlight/smoot-salute}.
  \end{itemize}
  
\end{frame}
% -----------------------------------------------------------------------------
\end{document}
