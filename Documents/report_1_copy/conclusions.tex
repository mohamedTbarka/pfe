\chapter{Conclusion}

We have described and proved correct a leader election algorithm for dynamic net- works. To provide for the temporal ordering of events that the algorithm requires, we use a generic notion of time–causal clocks–which can be implemented using, for instance, perfect clocks or logical clocks. Note that the algorithm is correct in the case of complete synchrony between clocks (perfect clocks) and also in the case of clocks with no bound on skew (logical clocks), but it is not correct for approximately synchronized clocks (which assume an upper bound on skew) unless they preserve causality. Notably, our definition of causal clocks does not include vector clocks (e.g., [8]), since vector clock values do not form a totally-ordered set in order to capture non-causality as well as causality1 An open question is how to extend our algorithm and its analysis to handle a wider range of clocks, such as approximately synchro- nized clocks and vector clocks. We identified different sets of circumstances under which the algorithm does not elect a leader unnecessarily. Depending on the types of clocks used to implement causal time and the amount of synchrony they provide, however, these circumstances tend to be different. It would be interesting to introduce different types of clocks, which not only preserve causality but also have some upper bound on skew, and see how they affect the stability condition of the algorithm. Moreover, an analysis of the time and message complexity needs to be performed, taking into account that using some clocks to implement causal time will be more efficient compared to others.

\paragraph{Acknowledgments}We thank Petr Kuznetsov, Ada Diaconescu and Sanaa EL FKIHI for helpful conversations, and the anonymous reviewers for comments that improved the presentation.

\begin{thebibliography}{1}
	
	\bibitem{1}
	B. Awerbuch, A. Richa, and C. Scheideler.
	\newblock A jamming-resistant MAC protocol for single-hop wireless networks.
	\newblock In {\em In Proc. 27th ACM Symp. on Principles of Distributed Computing}, pp. 45–54, 2008.
	
	
	\bibitem{2}
	J. Brunekreef, J.-P. Katoen, R. Koymans, and S. Mauw.
	\newblock Design and analysis of dynamic leader election protocols in broadcast networks.
	\newblock {\em Distributed Computing}, 9(4):157–171, 1996.
	
	\bibitem{3}
	O. Dagdeviren and K. Erciyes.
	\newblock A hierarchical leader election protocol for mobile ad hoc networks.
	\newblock {\em  In Proc. 8th Int’l Conf. on Computational Science, LNCS 5101}, pp. 509–518, 2008.
	
	\bibitem{4}
	Datta, A.K., Larmore, L.L., Piniganti, H.
	\newblock Self-stabilizing leader election in dynamic networks.
	\newblock {\em In: Proceedings of the 12th International Symposium on Stabilization, Safety, and Security of Distributed Systems}, pp. 35–49 (2010).
	
	\bibitem{5}
	A. Derhab and N. Badache.
	\newblock A self-stabilizing leader election algorithm in highly dynamic ad hoc mobile networks.
	\newblock {\em IEEE Trans. on Parallel and Distributed Systems}, 19(7):926–939, 2008.
	
	\bibitem{6}
	6. Dolev, S.
	\newblock Self-Stabilization. MIT Press, Cambridge, MA (2000)
	
	\bibitem{7}
	C. Fetzer and F. Cristian.
	\newblock A highly available local leader election service.
	\newblock {\em IEEE Trans. on Software Engineering}, 25(5):603–618, 1999.
	
	\bibitem{8}
	Fidge, C.
	\newblock Timestamps in message-passing systems that preserve the partial ordering.
	\newblock {Australian Computer Science Communications 10(1)}, 56–66 (1988).
	
	\bibitem{9}
	E. Gafni and D. Bertsekas.
	\newblock Distributed algorithms for generating loop-free routes in networks with frequently changing topology.
	\newblock {\em IEEE Trans. on Communications}, C-29(1):11–18, 1981.
	
	\bibitem{10}
	Z. Haas.
	\newblock  A new routing protocol for the reconfigurable wireless networks.
	\newblock In {\em Proc. 6th IEEE Int’l Conf. on Universal Personal Comm.}, pp. 562–566, 1997.
	
	\bibitem{11}
	S. Han and Y. Xia.
	\newblock  Optimal leader election scheme for peer-to-peer applications.
	\newblock In {\em Proc. 6th Int’l. Conf. on Networking}, page 29, 2007.
	
	\bibitem{12}
	K. Hatzis, G. Pentaris, P. Spirakis, V. Tampakis, and R. Tan.
	\newblock  Fundamental control algorithms in mobile networks.
	\newblock In {\em Proc. 11th ACM Symp. on Parallel Algorithms and Architectures}, pp. 251–260, 1999.
	
	\bibitem{13}
	Higham, L., Liang, Z.
	\newblock  Self-stabilizing minimum spanning tree construction on message-passing net- works.
	\newblock In {\em DISC01}, pp. 194–208 (2001).
	
	\bibitem{14}
	Howell, R.R., Nesterenko, M., Mizuno, M.
	\newblock Finite-state self-stabilizing protocols in message-passing systems.
	\newblock {\em Journal of Parallel and Distributed Computing 62(5)}, 792–817 (2002).
	
	\bibitem{15}
	R. Ingram, P. Shields, J. Walter, and J. Welch.
	\newblock An Asynchronous Leader Election Algorithm for Dynamic Networks. Technical Report 2009-1-1, Department of Computer Science and Engineering, Texas.
	
	\bibitem{16}
	L. Lamport.
	\newblock  Time, Clocks, and the Ordering of Events in a Distributed System.
	\newblock In {\em Communications of the ACM}, p.558, July 1978, Volume 21, Number 7.
	
	\bibitem{17}
	Lynch, N.A., Tuttle, M.R.
	\newblock An introduction to input/output automata. CWI-Quarterly 2(3), 219–246 (1989). Centrum voor Wiskunde en Informatica, Amsterdam, The Netherlands. Technical Memo MIT/LCS/TM-373, Laboratory for Computer Science, Massachusetts Institute of Technology, Cam- bridge, MA 02139, November 1988. Also, “Hierarchical Correctness Proofs for Distributed Algorithms,” in Proceedings of the Sixth Annual ACM Symposium on Principles of Distributed Computing, pages 137-151, Vancouver, British Columbia, Canada, August 1987.
	
	\bibitem{18}
	N. Malpani, J. Welch, and N. Vaidya.
	\newblock  Leader election algorithms for mobile ad hoc networks.
	\newblock In {\em Proc.ACM DIAL-M Workshop}, pp. 96–104, 2000.
	
	\bibitem{19}
	B. Mans and N. Santoro.
	\newblock  Optimal Elections in Faulty Loop Networks and Applications.
	\newblock {\em IEEE Trans. on Computers}, 47(3):286–297, 1998.
	
	\bibitem{20}
	S. Masum, A. Ali, and M. Bhuiyan.
	\newblock  Asynchronous leader election in mobile ad hoc networks.
	\newblock In {\em Proc. Int’l Conf. on Advanced Information Networking and Applications}, pp. 29–34, 2006.
	
	\bibitem{21}
	Y. Pan and G. Singh.
	\newblock A fault-tolerant protocol for election in chordal-ring networks with fail-stop processor failures.
	\newblock {\em IEEE Trans. on Reliability}, 46(1):11–17, 1997.
	
	\bibitem{22}
	V. Park and M. S. Corson.
	\newblock A highly adaptive distributed routing algorithm for mobile wireless networks.
	\newblock In {\em Proc. INFOCOM ’97}, pp. 1405–1413, 1997.
	
	\bibitem{23}
	P. Parvathipuram, V. Kumar, and G.-C. Yang.
	\newblock An efficient leader election algorithm for mobile ad hoc networks.
	\newblock In {\em Proc. 1st Int’l Conf. on Dist. Computing and Internet Technology, LNCS 3347}, pp. 32–41, 2004.
	
	\bibitem{24}
	M. Rahman, M. Abdullah-Al-Wadud, and O. Chae.
	\newblock Performance analysis of leader election algorithms in mobile ad hoc networks.
	\newblock {\em Int’l J. of Computer Science and Network Security}, 8(2):257–263, 2008.
	
	\bibitem{25}
	G. Singh.
	\newblock Leader Election in the Presence of Link Failures.
	\newblock {\em IEEE Trans. on Parallel and Distributed Systems}, 7(3):231–236, 1996.
	
	\bibitem{26}
	S. Stoller.
	\newblock Leader election in distributed systems with crash failures.
	\newblock Technical Report, Department of Computer Science, Indiana University, 1997.
	
	\bibitem{27}
	G. Tel.
	\newblock {\em G. Tel. Introduction to Distributed Algorithms, Second Edition.}
	\newblock Cambridge University Press, 2000.
	
	\bibitem{28}
	S. Vasudevan, J. Kurose, and D. Towsley.
	\newblock Design and analysis of a leader election algorithm for mobile ad hoc networks.
	\newblock In {\em In Proc.IEEE Int’l Conf. on Network Protocols}, pp. 350–360, 2004.
	
	\bibitem{29}
	Y. Wang and H. Wu.
	\newblock Replication-based efficient data delivery scheme for delay/fault-tolerant mobile sensor network (dft-msn).
	\newblock In {\em Proc. Pervasive Computing and Communications Workshop}, p. 5, 2006.
	
	\bibitem{30}
	R. Ingram, T. Radeva, P. Shields, S. Viqar, J.-E. Walter, and J.-L. Welch.
	\newblock A Leader Election Algorithm for Dynamic Networks with Causal Clocks
	\newblock , 2013.
	
	\bibitem{31}
	A. Casteigts.
	\newblock JBotSim: a Tool for Fast Prototyping of Distributed Algorithms in Dynamic Networks
	\newblock , 2010.
	
	\bibitem{32}
	JBOTSIM’s website: http://jbotsim.sf.net/.
	
	
	\bibitem{33}
	D. Angluin, J. Aspnes, Z. Diamadi, M. J. Fischer, and R. Per- alta.
	\newblock Computation in networks of passively mobile finite-state sensors
	\newblock, vol. 18, no. 4, pp. 235–253, 2006.
	
	\bibitem{34}
	M. Bastian, S. Heymann, and M. Jacomy, “Gephi.
	\newblock An open source software for exploring and manipulating networks.
	\newblock In {\em International AAAI conference on weblogs and social media}, vol. 2. AAAI Press Menlo Park, CA, 2009.
	
	\bibitem{35}
	M. Bauderon and M. Mosbah.
	\newblock A unified framework for de- signing, implementing and visualizing distributed algorithms.
	\newblock {\em Electr. Notes Theor. Comput. Sci.}, vol. 72, no. 3, pp. 13–24, 2003.
	
	\bibitem{36}
	Collective Authors.
	\newblock The NS-3 network simulator.
	\newblock , http:// www.nsnam.org/, 2009.
	
	\bibitem{37}
	B. Derbel.
	\newblock A Brief Introduction to ViSiDiA.
	\newblock , USTL, Tech. Rep., 2007.
	
	\bibitem{38}
	A. Dutot, F. Guinand, D. Olivier, and Y. Pigné.
	\newblock GraphStream: A Tool for bridging the gap between Complex Systems and Dynamic Graphs.
	\newblock {\em  EPNACS: Emergent Properties in Natural and Artificial Complex Systems}, 2007.
	
	\bibitem{39}
	A. Keränen, J. Ott, and T. Kärkkäinen.
	\newblock The one simulator for dtn protocol evaluation.
	\newblock In {\em IProceedings of the 2nd In- ternational Conference on Simulation Tools and Techniques}, ICST (Institute for Computer Sciences, Social-Informatics and Telecommunications Engineering), 2009, p. 55.

	\bibitem{40}
	I. Litovsky, Y. Métivier, and É. Sopena.
	\newblock Graph relabelling sys- tems and distributed algorithms.
	\newblock {\em Handbook of graph gram- mars and computing by graph transformation}, vol. 3, pp. 1–56, 1999.

	\bibitem{41}
	A. Varga et al.
	\newblock The OMNeT++ discrete event simulation system.
	\newblock In {\em  Proceedings of the European Simulation Multi-conference (ESM’01)}, 2001, pp. 319–324.
	
	\bibitem{42}
	JBOTSIM’s website: http://jbotsim.sf.net/. 
	
	\bibitem{43}
	D. Angluin, J. Aspnes, Z. Diamadi, M. J. Fischer, and R. Peralta, “Computation in networks of passively mobile finite-state sensors,” Distributed Computing, vol. 18, no. 4, pp. 235–253, 2006.
	
	\bibitem{44}
	M. Bastian, S. Heymann, and M. Jacomy, “Gephi: An open source software for exploring and manipulating networks,” in International AAAI conference on weblogs and social media, vol. 2. AAAI Press Menlo Park, CA, 2009.
	
	\bibitem{45}
	M. Bauderon and M. Mosbah, “A unified framework for de- signing, implementing and visualizing distributed algorithms,” Electr. Notes Theor. Comput. Sci., vol. 72, no. 3, pp. 13–24, 2003.
	
	\bibitem{46}
	Collective Authors, “The NS-3 network simulator,” http:// www.nsnam.org/, 2009.
	
	\bibitem{47}
	B. Derbel, “A Brief Introduction to ViSiDiA,” USTL, Tech. Rep., 2007.
	
	\bibitem{48}
	A. Dutot, F. Guinand, D. Olivier, and Y. Pigné, “GraphStream: A Tool for bridging the gap between Complex Systems and Dynamic Graphs,” EPNACS: Emergent Properties in Natural and Artificial Complex Systems, 2007.
	
	\bibitem{49}
	A. Keränen, J. Ott, and T. Kärkkäinen, “The one simulator for dtn protocol evaluation,” in Proceedings of the 2nd International Conference on Simulation Tools and Techniques. ICST (Institute for Computer Sciences, Social-Informatics and Telecommunications Engineering), 2009, p. 55.
	
	\bibitem{50}
	I. Litovsky, Y. Métivier, and É. Sopena, “Graph relabelling systems and distributed algorithms,” Handbook of graph gram- mars and computing by graph transformation, vol. 3, pp. 1–56, 1999. ]
	
	\bibitem{51}
	A. Varga et al., “The OMNeT++ discrete event simulation system,” in Proceedings of the European Simulation Multi-conference (ESM’01), 2001, pp. 319–324.
	
\end{thebibliography}