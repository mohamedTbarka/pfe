\chapter{Implementation}
\section{JBotSim: a Tool for Fast Prototyping of Distributed Algorithms in Dynamic Networks}
\subsection{What's JBotSim ?}
JBotSim is an open source simulation library that is dedicated to distributed algorithms in dynamic networks. I developed it with the purpose in mind to make it possible to implement an algorith- mic idea in minutes and interact with it while it is running (e.g., add, move, or delete nodes). Besides interaction, JB OT S IM can also be used to prepare live demos of an algorithm and to show it to colleagues or students, as well as to assess the algorithm per- formance. JB OT S IM is not a competitor of mainstream simulators such as NS3 [5], OMNet [10], or The One [8], in the sense that it does not aim to implement real-world networking protocols. Quite the opposite, JB OT S IM aims to remain technology-insensitive and to be used at the algorithmic level, in a way closer in spirit to the ViSiDiA project (a general-purpose platform for distributed algo- rithms). Unlike ViSiDiA, however, JB OT S IM natively supports mobility and dynamic networks (as well as wireless communica- tion). Another major difference with the above tools is that it is a library rather than a software: its purpose is to be used in other pro- grams, whether these programs are simple scenarios of full-fledged software. Finally, JB OT S IM is distributed under the terms of the LGPL licence, which makes it easily extensible by the community. Whether the algorithms are centralized or distributed, the natu- ral way of programming in JB OT S IM is event-driven: algorithms are specified as subroutines to be executed when particular events occur (appearance or disappearance of a link, arrival of a message, clock pulse, etc.). Movements of the nodes can be controlled ei- ther by program or by means of live interaction with the mouse (adding, deleting, or moving nodes around with left-click, right- click, or drag and drop, respectively). These movements are typi- cally performed while the algorithm is running, in order to visualize it or test its behavior in challenging configurations. The present document offers a broad view of JB OT S IM’s main features and design traits. I start with preliminary information in
\subsection{Practical Aspects}
\paragraph{}In this short section, I show how to install JB OT S IM and run it with a first basic example (this step is not required to keep reading the present document). I also provide links to online documentation and examples, for readers who would like to explore JB OT S IM’s features beyond this paper.
\subsubsection{Fetching JBOTSIM}
