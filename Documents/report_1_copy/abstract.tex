\begin{abstract}
A hierarchical algorithm for electing a leaders' hierarchy in an asynchronous network with dynamically changing communication topology is presented including a remoteness's constraint towards each leader. The algorithm ensures that, no matter what pattern of topology changes occur, if topology changes cease, then eventually every connected component contains a unique leaders' hierarchy. The algorithm combines ideas from the Temporally Ordered Routing Algorithm (TORA) for mobile ad hoc networks with a wave algorithm, all within the framework of a height-based mechanism for reversing the logical direction of communication links. Moreover, an improvement from the algorithm in is the introduction of logical clocks as the nodes’ measure of time, instead of requiring them to have access to a common global time. This new feature makes the algorithm much more flexible and applicable to real situations, while still providing a correctness proof. It is also proved that in certain well behaved situations, a new leader is not elected unnecessarily.
\end{abstract}


\newpage

\renewcommand{\abstractname}{Résumé}

\begin{abstract}
Un algorithme hiérarchique pour l'élection d'une hiérarchie de leaders dans un réseau asynchrone avec une topologie de comunication en évolution dynamique est présenté, incluant une contrainte d'éloignement vis-à-vis de chaque leader. L'algorithme garantit que, quel que soit le modèle de modification de la topologie utilisé, si cette modification cesse, tous les composants connectés contiennent une hiérarchie de leaders unique. L'algorithme combine des idées de l'algorithme TORA (Temporally Ordered Algorithm) pour les réseaux ad hoc mobiles avec un algorithme d'onde, le tout dans le cadre d'un mécanisme basé sur la hauteur permettant d'inverser le sens logique des liaisons de communication. De plus, une amélioration de l’algorithme est l’introduction d’horloges logiques comme mesure du temps des noeuds, au lieu de leur demander d’avoir accès à une heure globale commune. Cette nouvelle fonctionnalité rend l'algorithme beaucoup plus flexible et applicable à des situations réelles, tout en fournissant une preuve de correction. Il est également prouvé que dans certaines situations bien conduites, un nouveau chef n’est pas élu inutilement.
algorithme
Translations of algorithme
NounFrequency
algorithm
algorithme
Definitions of algorithme
Noun
1
Ensemble de règles définies en vue d’obtenir un résultat déterminé.
Synonyms of algorithme
Noun
algorithmique

\end{abstract}


