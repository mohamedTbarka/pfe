\chapter{Preliminaries}
\section{System Model}
\paragraph{}We assume a system consisting of a set P of computing nodes and a set $\chi$ of directed communication channels from one node to another node. $\chi$ consists of one channel for each ordered pair of nodes, i.e., every possible channel is represented. The nodes are assumed to be completely reliable. The channels between nodes go up and down, due to the movement of the nodes. While a channel is up, the communication across it is in first-in-first-out order and is reliable but asynchronous (see below for more details).
\paragraph{}We model the whole system as a set of (infinite) state machines that interact through shared events (a specialization of the IOA model [17]). Each node and each channel is modeled as a separate state machine. The events shared by a node and one of its outgoing channels are notifications that the channel is going up or going down and the sending of a message by the node over the channel; the channel up/down notifications are initiated by the channel and responded to by the node, while the message sends are initiated by the node and responded to by the channel. The events shared by a node and one of its incoming channels are notifications that a message is being delivered to the node from the channel; these events are initiated by the channel and responded to by the node.
\section{Modeling Asynchronous Dynamic Links}
\paragraph{}We now specify in more detail how communication is assumed to occur over the dynamic links. The state of $Channel(u, v)$, which models the communication channel from node $u$ to node $v$, consists of a statusuv variable and a queue $mqueueuv$ of messages.
\paragraph{}The possible values of the statusuv variable are Up and Down. The channel transitions between the two values of its $status_uv$ variable through $ChannelUp_uv$ and $ChannelDown_uv$ events, called the “topology change” events. We assume that the $ChannelUp$ and $ChannelDown$ events for the channel alternate. The $ChannelUp$ and $ChannelDown$ events for the channel from $u$ to $v$ occur simultaneously at node $u$ and the channel, but do not occur at node $v$.
\paragraph{}The variable $mqueue_uv$ holds messages in transit from $u$ to $v$. An attempt by node $u$ to send a message to node $v$ results in the message being appended to $mqueue_uv$ if the channel’s status is $Up$; otherwise there is no effect. When the channel is $Up$, the message at the head of $mqueue_uv$ can be delivered to node $v$; when a message is delivered, it is removed from $mqueue_uv$ . Thus, messages are delivered in FIFO order.
\paragraph{}When a $ChannelDown_uv$ event occurs, $mqueue_uv$ is emptied. Neither $u$ nor $v$ is alerted to which messages in transit have been lost. Thus, the messages delivered to node $v$ from node $u$ during a (maximal-length) interval when the channel is $Up$ form a prefix of the messages sent by node $u$ to node $v$ during that interval.
\section{Configurations and Executions}
The notion of configuration is used to capture an instantaneous snapshot of the state of the entire system. A configuration is a vector of node states, one for each node in $\mathcal{P}$, and a vector of channel states, one for each channel in $\chi$ . In an initial configuration:
\begin{list}{--}{spacing}
	\item each node is in an initial state (according to its algorithm),
	\item for each channel $Channel(u, v)$, $mqueueuv$ is empty, and
	\item for all nodes $u$ and $v$, $statusuv = statusvu$ (i.e., either both channels between $u$ and $v$ are up, or both are down).
\end{list}
Define an execution as an infinite sequence $C_0, e_1 ,C_1, e_2 ,C_2,...$ of alternating configurations and events, starting with an initial configuration and, if finite, ending with a configuration such that the sequence satisfies the following conditions:
\begin{list}{--}{spacing}
\item  $C_0$ is an initial configuration.
\item The preconditions for event ei are true in $C_{i-1}$ for all $i\geq 1$.
\item $C_i$ is the result of executing event $e_i$ on configuration $C_{i-1}$ , for all $i\geq 1$ (only the node and channel involved in an event change state, and they change according to their state machine transitions).
\item If a channel remains Up for infinitely long, then every message sent over the channel during this Up interval is eventually delivered.
\item For all nodes u and v, Channel(u, v) experiences infinitely many topology change events if and only if Channel(v, u) experiences infinitely many topology change events; if they both experience finitely many, then after the last one, statusuv = statusvu.
\end{list}
\paragraph{}Given a configuration of an execution, define an undirected graph Gchan as fol- lows: the vertices are the nodes, and there is an (undirected) edge between vertices u and v if and only if at least one of $Channel_{uv}$ and $Channel_{vu}$ is Up. Thus Gchan indicates all pairs of nodes u and v such that either u can send messages to v or v can send messages to u. If the execution has a finite number of topology change events, then Gchan never changes after the last such event, and we denote the final version of final f inal Gchan as Gchan . By the last bullet point above, an edge in Gchan indicates bidirectional communication ability between the two endpoints.
\paragraph{}We also assign a positive real-valued global time $gt$ to each event $e_i$ , $i \geq 1$, such that $gt(ei) < gt(e_{i+1})$ and, if the execution is infinite, the global times increase without bound. Each configuration inherits the global time of its preceding event, so $gt(C_i) = gt(e_i)$ for $i \geq 1$; we define $gt(C_0)$ to be $0$. We assume that the nodes do not have access to $gt$.
\paragraph{}Instead, each node u has a causal clock $\mathcal{T}_u$ , which provides it with a non-negative real number at each event in an execution. Tu is a function from global time (i.e., positive reals) to causal clock times; given an execution, for convenience we some- times use the notation Tu (ei ) or Tu (Ci ) as shorthand for Tu (gt(ei )) or Tu (gt(Ci )). The key idea of causal clocks is that if one event potentially can cause another event, then the clock value assigned to the first event is less than the clock value assigned to the second event. We use the notion of happens-before to capture the concept of potential causality. Recall that an event e1 is defined to happen before [16] another event e2 if one of the following conditions is true:
\begin{enumerate}
	\item Both events happen at the same node, and $e_1$ occurs before $e_2$ in the execution.
	\item $e_1$ is the send event of some message from node $u$ to node $v$, and $e_2$ is the receive event of that message by node $v$.
	\item There exists an event $e$ such that $e_1$ happens before $e$ and $e$ happens before $e_2$.
\end{enumerate}
The causal clocks at all the nodes, collectively denoted T , must satisfy the following properties:
\begin{list}{--}{spacing}
	\item For each node $u$, the values of \mathcal{T}_u are increasing, i.e., if $e_i$ and $e_j$ are events involving $u$ in the execution with $i < j$, then $\mathcal{T}_u(e_i) < \mathcal{T}_u(e_j)$. In particular, if there is an infinite number of events involving $u$, then Tu increases without bound.
	\item $\mathcal{T}$ preserves the happens-before relation [16] on events; i.e., if event $e_i$ happens before event $e_j$, then $\mathcal{T}(e_i)$ < $\mathcal{T}(e_j)$.
\end{list}
Our description and proof of the algorithm assume that nodes have access to causal clocks. One way to implement causal clocks is to use perfect clocks, which ensure that Tu (t) = t for each node u and global time t. Since an event that causes an- other event must occur before it in real time, perfect clocks capture causality. Perfect clocks could be provided by, say a GPS service, and were assumed in the prelimi- nary version of this paper [15]. Another way to implement causal clocks is to use Lamport’s logical clocks [16], which were specifically designed to capture causality.
\section{Problem Definition}
\paragraph{}Each node u in the system has a local variable lidu to hold the identifier of the node currently considered by u to be the leader of the connected component containing u.
\paragraph{}In every execution that includes a finite number of topology change events, we require that the following eventually holds: Every connected component CC of the f inal final topology graph Gchan contains a node ℓ, the leader, such that lidu = ℓ for all nodes u ∈ CC, including ℓ itself.