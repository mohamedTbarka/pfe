% On découpe ce document complexe en plusieurs sous-fichiers séparés.
% Cela permettra notamment de réarranger les transparents facilement 
% lors de l'élaboration du document.

% La définition de la classe beamer avec tous les styles afférents

\RequirePackage{currfile} 

\documentclass{beamer}

\input{preambule/special_beamer.tex}

% Les autres packages utiles  notamment pour le français, les accents ou Python
\input{preambule/autres_packages.tex}
% Les macros et raccourcis personnels
\input{preambule/macros.tex}

% On définit le titre et l'auteur du document

% L'argument optionnel (entre crochets) donne le titre qui sera mis sur chaque slide
\title[TIPE Mercure]{TIPE: Précession du Périhélie de Mercure}
\author{Jean-Julien \textsc{Fleck}} % Votre nom
% L'épreuve b(car on n'a pas le droit de signaler sa provenance à un concours) (là encore, l'argument optionnel apparaît sur chaque slide)
%\institute()[TIPE]{Épreuve de TIPE}
\date{Session 2019} 

% On démarre le document proprement dit
\begin{document}

% La page de titre et la table des matières
\input{slides/00_titre_et_tdm.tex}

% La première grande partie: introduction du sujet
% Titre de la premiere partie
\section{Introduction Historique}

%%%%%%%%%%%%%%%%%%%%%%%%%%%%%%%%%%%%%%%%%%%%%%%%
% Première diapo
%%%%%%%%%%%%%%%%%%%%%%%%%%%%%%%%%%%%%%%%%%%%%%%%
\begin{frame}
\frametitle{Introduction historique}
\framesubtitle{Position du problème}

\begin{itemize}
	\item	<1->	La mécanique newtonienne, une mécanique bien huilée
	\item	<2->	Précession du périhélie de Mercure
	
	\visible<3->{
	\begin{figure}
	\includegraphics[width=0.8\linewidth]{figures/Fig01}
	\end{figure}
	}
	
\end{itemize}

\end{frame}


%%%%%%%%%%%%%%%%%%%%%%%%%%%%%%%%%%%%%%%%%%%%%%%%
% Deuxième diapo
%%%%%%%%%%%%%%%%%%%%%%%%%%%%%%%%%%%%%%%%%%%%%%%%
\begin{frame}
\frametitle{Introduction historique}
\framesubtitle{Explication classique}

\begin{itemize}
	\item	<1->	Théorie des perturbations: explication de $531''$ d'arc/siècle
	\item	<2->	Vulcain, un bon candidat (Le Verrier, 1859)
	\item	<3->	mais pas de confirmation expérimentale...
	
\end{itemize}

\end{frame}


%%%%%%%%%%%%%%%%%%%%%%%%%%%%%%%%%%%%%%%%%%%%%%%%
% Troisième diapo
%%%%%%%%%%%%%%%%%%%%%%%%%%%%%%%%%%%%%%%%%%%%%%%%
\begin{frame}
\frametitle{Introduction historique}
\framesubtitle{Complément relativiste}

\begin{itemize}
	\item	<1->	La relativité restreinte permet déjà un mieux ($7''$ d'arc 
	en plus par siècle)

	\item	<2->	mais c'est la générale qui va sauver la mise en ajoutant 
	les $43''$ qui manquaient
	
\end{itemize}

\end{frame}


% La 2e partie: Le point de vue de la relativité restreinte
% Titre de la partie
\section{Le point de vue de la Relativité Restreinte}

%%%%%%%%%%%%%%%%%%%%%%%%%%%%%%%%%%%%%%%%%%%%%%%%
% Première diapo (avec des équations)
%%%%%%%%%%%%%%%%%%%%%%%%%%%%%%%%%%%%%%%%%%%%%%%%
\begin{frame}
\frametitle{Relativité Restreinte}
\framesubtitle{Dynamique relativiste}

\begin{itemize}
	\item	<1->	Nouvelle définition de la quantité de mouvement
	
	\onslide <2->{
	$$\vec{p} = \gamma\,m\vec{v}
	\qquad\text{avec}\qquad
	\gamma = \frac{1}{\sqrt{1 - \frac{v^2}{c^2}}}
	$$
	}
	
	\item	<3->	Avec les développements \ofg{classiques} et en posant 
	$u=1/r$, on en arrive à l'équation
	
	\onslide <4->{
	$$
	\onslide <5->
	\underbrace{
	\onslide <4->
	    \frac{\dd^2 u}{\dd \theta^2} + u = \frac{G M m\, E}{L^2\, c^2}
	\onslide <5->
	    }_{
	    \text{Partie usuelle}}
	\onslide <4->
	            + 
	\onslide <6->
	\underbrace{
	\onslide <4->	
	            \frac{\pa{G M m}^2}{L^2\, c^2}\, u
    \onslide <6->
               }_{\text{Partie relativiste}}
	$$
	}
	
	
\end{itemize}

\end{frame}


%%%%%%%%%%%%%%%%%%%%%%%%%%%%%%%%%%%%%%%%%%%%%%%%
% Deuxième diapo
%%%%%%%%%%%%%%%%%%%%%%%%%%%%%%%%%%%%%%%%%%%%%%%%
\begin{frame}
\frametitle{Relativité Restreinte}
\framesubtitle{Équation de l'ellipse}

\begin{itemize}
	\item	<1->	Équation différentielle remise en forme
	
	\onslide <2->{
	$$
	    \frac{\dd u}{\dd \theta} + B^2\, u = A 
	\onslide <3->
	        \quad \text{avec} \quad 
	            B  = \sqrt{1 - \pa{\frac{G M m}{L\, c}}^{\!\!2} }
	$$
	}
	
	\item	<4->	Équation de l'\ofg{ellipse}
	
	\onslide <5->
	$$
	u = \frac{A}{B^2} \pa{1 + e\cos{\pac{B\pa{\theta - \theta_0}}}}
	\onslide <6->
	\quad\text{soit}\quad
	r = \frac{p}{1+e\cos{\pac{B\pa{\theta - \theta_0}}}}
	$$
	
\end{itemize}

\end{frame}

%%%%%%%%%%%%%%%%%%%%%%%%%%%%%%%%%%%%%%%%%%%%%%%%
% Troisième diapo
%%%%%%%%%%%%%%%%%%%%%%%%%%%%%%%%%%%%%%%%%%%%%%%%
\begin{frame}
\frametitle{Relativité Restreinte}
\framesubtitle{Avance du périhélie}

\begin{itemize}
	\item	<1->	L'\ofg{ellipse} ne se referme pas sur elle-même du fait que $B\neq1$
	\item	<2->	Entre deux périhélies successifs, $\theta$ tourne de $2\pi +\delta$ où
	\onslide<3->
	$$
	\boxed{
    \delta = 2\pi\pa{\frac{1}{B} - 1}
    \approx \pi \pa{\frac{G\, M\, m}{L\, c}}^{\!\!2} 
    = \pi \, \frac{G M}{p\, c^2}
    = \pi\, \frac{G M}{a\, c^2\, \pa{1-e^2}}	
    }
	$$
	
	\item	<4-> Malheureusement, l'application numérique ne donne \ofg{que} 
	$7''$ d'arc par siècle...
	
\end{itemize}

\end{frame}


% La 3e partie: Le point de vue de la relativité générale
% Le titre de la partie
\section{Le point de vue de la Relativité Générale}

%%%%%%%%%%%%%%%%%%%%%%%%%%%%%%%%%%%%%%%%%%%%%%%%
% Première diapo
%%%%%%%%%%%%%%%%%%%%%%%%%%%%%%%%%%%%%%%%%%%%%%%%

\begin{frame}
\frametitle{Relativité Générale}
\framesubtitle{Et zut...}

\begin{itemize}
	\item	<1->	Il n'y a plus guère de temps pour parler
	\item	<2->	Alors zou, on balance le résultat des calculs

$$
    \onslide<3->
    \underbrace{
    \onslide<2->    
    \frac{\dd^2 u}{\dd \theta^2} + u 
        = 
        \frac{G M m^2}{L^2} 
    \onslide<3->
    }_{\text{Partie classique}}
    \onslide<2->
    + 
    \onslide<3->
    \underbrace{
    \onslide<2->    
    \frac{3\, G M}{c^2}\, u^2
    \onslide<3->
    }_{\text{RG}}
    \onslide<2->
$$

	\item	<4->	Après avoir bien bossé, on obtient finalement
\onslide<4->
$$
	\boxed{
		\delta = 6\, \pi \, \frac{G M}{a\, c^2\pa{1-e^2}}
			}
$$

\end{itemize}

\end{frame}


%%%%%%%%%%%%%%%%%%%%%%%%%%%%%%%%%%%%%%%%%%%%%%%%
% Deuxième diapo: le code informatique impose un 
% environnement "fragile" pour la frame
%%%%%%%%%%%%%%%%%%%%%%%%%%%%%%%%%%%%%%%%%%%%%%%%

\begin{frame}[fragile]
\frametitle{Relativité Générale}
\framesubtitle{Le code informatique}

\begin{code}
\begin{minted}[linenos]{python}

import scipy as sp
import scipy.optimize

def ma_fonction(x):
    return []

# À vous de remplir les choses adéquates...

\end{minted}
\end{code}
\end{frame}


% Conclusion
\input{slides/04_conclusion.tex}

% Les slides d'exemples 
% (à commenter si bien sûr vous n'en voulez pas... 
% ils sont juste là pour servir d'exemples de base)
\section{Exemples divers}
\input{slides/exemples_listes_a_apparitions_successives.tex}
\input{slides/exemples_figure.tex}
\input{slides/exemples_equation.tex}

\end{document}


