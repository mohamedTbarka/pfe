\chapter{Correctness}
In this section, we show that, once topology changes
cease, the algorithm eventually terminates with each con-
nected component forming a leader-oriented DAG. First, we
make some definitions regarding the information concerning nodes’ heights that exists in the system and prove some
properties about it. Then we prove that, after the last topology change, each node elects itself a finite number of times
and a finite number of new reference levels are started. As
a result, we show that eventually no messages are in transit
and at that point we have a leader-oriented DAG.
Throughout the proof, consider an arbitrary execution of
the algorithm in which the last topology change occurs at
some time $t_{LTC}$ , and consider any connected component of
the final topology.

\section{Height Tokens and Their Properties}
\textbf{Property A :} If h is a height token for a node v in the (v, u)
height sequence, then:
(i) nlts(h) ≤ LC v and τ (h) ≤ LC v
(ii) If h is in transit, then
nlts(h) ≤ T and τ (h) ≤ T
where T is the logical timestamp on the Update mes-
sage from v to u containing h.
(iii) If h is in u’s height array then
nlts(h) ≤ LC u and τ (h) ≤ LC u .

\textbf{Property B:} Let h 0 , h 1 , . . . , h m be the (u, v) height sequence
for any Link{u, v} whose status is Up or ComingUp u . Then
the following are true:
(1) h 0 = h 1 .
(2) For all l, 0 ≤ l < m, LP(h l ) ≤ LP(h l+1 ).
(3) For all l, 0 ≤ l < m, if LP(h l ) = LP(h l+1 ), then
RL(h l ) ≥ RL(h l+1 ).

\section{Bounding the Number of Elections}
\textbf{Property C:} For each height token h with RL (t, p, r), either
t = p = r = 0, or t > 0, p is a node id, and r is 0 or 1.

\textbf{Property D:} Let h be a height token for some node u. If
LP(h) = (−s, l), where LC l (t) = s at time t and t ≥ t LTC ,
then RL(h) = (0, 0, 0) and δ (h) is the distance in LT (−s, l)
from l to u.


\textbf{Lemma 1:}Any node u that adopts leader pair (−s, l) for
any l and any s, where LC l (t) = s and t > t LTC , never sub-
sequently becomes a sink.

\textbf{Lemma 2:}No node elects itself more than a finite number
of times after t LTC.

\section{Bounding the Number of New Reference Levels}
\textbf{Property E:} If h and h ′ are two height tokens for the same
node u with RL(h) = RL(h ′ ) and LP(h) = LP(h ′ ), then
δ (h) = δ (h ′ ).
\textbf{Property F:} If there is a height token for node u with RL
prefix (t, p), where LC u (t ′ ) = t and t ′ ≥ t LTC , then u is in
RD(t, p).
\textbf{Property G:} If there is a height token for node u with RL
(t, p, 1), where LC u (t ′ ) = t and t ′ ≥ t LTC , then all neighbors
of u are in RD(t, p).
\textbf{Property H:} Suppose node u has height h u , neighboring
node v has height h v , and u’s view of v’s height is h ′ v , all
with the same LP. If h u < h ′ v , then h ′ v = h v .
Property I: Consider two height tokens, h u for a node u
with RL(h u ) = (t, p, r u ) and δ (h u ) = d u , and h v for a neigh-
boring node v with RL(h v ) = (t, p, r v ) and δ (h v ) = d v , where
LC p (t ′ ) = t and t ′ ≥ t LTC . Then the following are true:
(1) r u ≤ r v if and only if u is a predecessor of v in RD(t, p).
(2) If r u = r v = 0, then d u > d v if and only if u is a prede-
cessor of v.
(3) If r u = r v = 1, then d v > d u if and only if u is a prede-
cessor of v.
Lemma 3 Every node starts a finite number of new RLs
after t LTC .
\section{Bounding the Number of Messages}
\textbf{Lemma 4:}Eventually every node in the connected compo-
nent has the same leader pair.

\textbf{Lemma 5:} Eventually there are no messages in transit.

\textbf{Lemma 6:} Eventually every node has an accurate view of
its neighbors’ heights.
\section{Leader-Oriented DAG}
\textbf{Property J:} A node is never a sink in its own view.
\textbf{Property K:} Consider any height token h for node u. If
RL(h) = (0, 0, 0), then δ (h) ≥ 0. Furthermore, δ (h) = 0 if
and only if u is a leader.

{Theorem 7} Eventually the connected component is a
leader-oriented DAG.




